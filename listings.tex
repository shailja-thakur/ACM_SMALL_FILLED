% % Improved LaTeX listings for AAAI submission
% % Add these packages to your preamble if not already included:
% % \usepackage{tcolorbox}
% % \usepackage{xcolor}
% % \usepackage{listings}
% % \usepackage[dvipsnames]{xcolor}
% % \usepackage[most]{tcolorbox}
% % \usepackage{enumitem}

% % --- CUSTOM COLOR DEFINITIONS ---


% % Define colors for code-like appearance
% \definecolor{promptblue}{RGB}{0, 102, 204}
% \definecolor{lightgray}{RGB}{245, 245, 245}
% \definecolor{darkgray}{RGB}{64, 64, 64}
% \definecolor{commentgreen}{RGB}{0, 128, 0}
% \definecolor{keywordblue}{RGB}{0, 0, 255}
% \definecolor{stringred}{RGB}{163, 21, 21}

% \begin{figure}[t!]
% \centering

% \begin{tcolorbox}[
%     enhanced,
%     fontupper=\ttfamily\scriptsize,
%     colback=lightgray,
%     colframe=darkgray,
%     boxrule=0.8pt,
%     arc=3pt,
%     enlarge top by=0.5mm,
%     enlarge bottom by=0.5mm,
%     title={\scriptsize\textbf{Forward Chain-of-Thought Sample}},
%     coltitle=white,
%     colbacktitle=darkgray,
%     boxsep=2pt,
%     left=4pt,
%     right=4pt,
%     top=2pt,
%     bottom=2pt,
%     nobeforeafter,
%     before upper={\setlength{\parskip}{1pt}},
% ]

% {\color{promptblue}\textbf{<Instruction>}}

% You are given a Python function. Predict what the function will output given the input.

% {\color{promptblue}\textbf{<Function>}}

% [Function Signature and Docstring]

% {\color{promptblue}\textbf{<Question>}}

% What is the output of `function\_name` given the input?

% {\color{promptblue}\textbf{<CoT>}}
% \begin{enumerate}[leftmargin=8pt, itemsep=1pt, parsep=0pt]
% \item The function begins by initializing local variables (e.g., `counter=0`, `result=[]`).
% \item Based on the input `...`, the program enters a loop/conditional branch.  
% \item The state of variables is updated at each step, reflecting the execution trace.
% \item The final return value is computed based on the terminal state of the variables.
% \end{enumerate}

% {\color{promptblue}\textbf{<Predicted Output>}}

% [Final Output]

% \end{tcolorbox}
% {\footnotesize\textbf{(a) A forward sample reasons from input to output.}}

% \begin{tcolorbox}[
%     enhanced,
%     fontupper=\ttfamily\scriptsize,
%     colback=lightgray,
%     colframe=darkgray,
%     boxrule=0.8pt,
%     arc=3pt,
%     enlarge top by=0.5mm,
%     enlarge bottom by=0.5mm,
%     title={\scriptsize\textbf{Backward Chain-of-Thought Sample}},
%     coltitle=white,
%     colbacktitle=darkgray,
%     boxsep=2pt,
%     left=4pt,
%     right=4pt,
%     top=2pt,
%     bottom=2pt,
%     nobeforeafter,
%     before upper={\setlength{\parskip}{1pt}},
% ]

% {\color{promptblue}\textbf{<Instruction>}}

% You are given a Python function and its output. Predict what input could have produced this output.

% {\color{promptblue}\textbf{<Function>}}

% [Function Signature and Docstring]

% {\color{promptblue}\textbf{<Question>}}

% Given the function and the output, what input could have produced it?

% {\color{promptblue}\textbf{<CoT>}}
% \begin{enumerate}[leftmargin=8pt, itemsep=1pt, parsep=0pt]
% \item To obtain the output `...`, the function must have returned from line `L`.
% \item This requires the variable `...` to hold the value `...` at that point.
% \item Tracing backwards, to achieve this state, the preceding loop must have executed `N` times.
% \item For the loop to execute `N` times, the initial input must satisfy the condition `...`.
% \end{enumerate}

% {\color{promptblue}\textbf{<Plausible Input>}}

% [Predicted Input]

% \end{tcolorbox}
% {\footnotesize\textbf{(b) A backward sample reasons to plausible input.}}

% \caption{Abstract templates for the generated bi-directional CoT data.}
% \label{fig:cot-templates}
% \end{figure}






% Improved LaTeX listings for AAAI submission
% Add these packages to your preamble if not already included:
% \usepackage{tcolorbox}
% \usepackage{xcolor}
% \usepackage{listings}
% \usepackage[dvipsnames]{xcolor}
% \usepackage[most]{tcolorbox}
% \usepackage{enumitem}

% --- CUSTOM COLOR DEFINITIONS ---

 % \begin{table*}[t!]
  % \centering
  % \caption{Comprehensive evaluation results across models, datasets, and training configurations on CruxEval and LiveCodeBench benchmarks. The experiments follow a top-down approach, with the winning configuration
  % from
  %  one stage used in the next. Best results for each stage are highlighted in \textbf{bold}.}
  % \label{tab:comprehensive-results}
  % \resizebox{\textwidth}{!}{%
  % \begin{tabular}{l|l|l|c|c|c|cc|cc}
  % \hline
  % \multirow{2}{*}{\textbf{Experiment Stage}} & \multirow{2}{*}{\textbf{Model}} & \multirow{2}{*}{\textbf{Training Config}} & \multirow{2}{*}{\textbf{Data Subset}} & \multirow{2}{*}{\textbf{Data Config}} &
  % \textbf{LiveCodeBench-Exec} &
  % \multicolumn{2}{c|}{\textbf{CruxEval Output}} & \multicolumn{2}{c}{\textbf{CruxEval Input}} \\
  %  & & & & & \textbf{Pass@1} & \textbf{Pass@1} & \textbf{Pass@5} & \textbf{Pass@1} & \textbf{Pass@5} \\
  % \hline
  % \hline
  % % \multicolumn{9}{c}{\textit{Qwen2.5-7B Model Family}} \\
  % % \hline
  % \multirow{2}{*}{Baselines} & \multirow{2}{*}{Qwen2.5-7B} & Base (Pre-trained) & N/A & N/A & -- & -- & -- & -- & -- \\
  %  & & Instruct (General) & Standard & N/A & 21.7 & 32.5 & 40.6 & -- & -- \\
  % \hline
  % \multirow{3}{*}{1. Data Curation} & \multirow{6}{*}{Qwen2.5-7B} & \multirow{3}{*}{Fwd} & 18k & Difficulty-Filt. & 66.9 \textcolor{green}{(+45.2)} & 58.4 \textcolor{green}{(+25.9)} & 75.5 \textcolor{green}{(+34.9)} &
  % 13.5 & 27.4 \\
  %  & & & 25k (best perf.) & Correctness-Filt. & \textbf{67.0}$^{\dagger}$ \textcolor{green}{(+45.3)} & 57.5 \textcolor{green}{(+25.0)} & 73.9 \textcolor{green}{(+33.3)} & -- & -- \\
  %  & & & 54k & Full Set & 66.5 \textcolor{green}{(+44.8)} & \textbf{58.6}$^{\dagger}$ \textcolor{green}{(+26.1)} & \textbf{76.0}$^{\dagger}$ \textcolor{green}{(+35.4)} & -- & -- \\
  % \cline{1-1}\cline{3-10}
  % \multirow{3}{*}{2. Reasoning Direction} & & Fwd & TBD & TBD & -- & 57.5 & 73.9 & -- & -- \\
  %  & & Bwd & TBD & TBD & -- & 50.4 & 69.8 & -- & -- \\
  %  & & Bi-directional & TBD & TBD & -- & 59.7 & 75.4 & -- & -- \\
  % \hline
  % \hline
  % % \multicolumn{9}{c}{\textit{Granite-3.3-8B Model Family}} \\
  % % \hline
  % \multirow{2}{*}{Baselines} & \multirow{2}{*}{Granite-3.3-8B} & Base (Pre-trained) & N/A & N/A & -- & -- & -- & -- & -- \\
  %  & & Instruct (General) & Standard & N/A & -- & 21.8 & 32.9 & -- & -- \\
  % \hline
  % \multirow{3}{*}{1. Data Curation} & \multirow{6}{*}{Granite-3.3-8B} & \multirow{3}{*}{Fwd} & 18k & Difficulty-Filt. & 43.5 & 36.1 \textcolor{green}{(+14.3)} & 58.2 \textcolor{green}{(+25.3)} & 35.8 & 57.9 \\
  %  & & & 25k (best perf.) & Correctness-Filt. & \textbf{44.9}$^{\dagger}$ & \textbf{42.7}$^{\dagger}$ \textcolor{green}{(+20.9)} & \textbf{64.7}$^{\dagger}$ \textcolor{green}{(+31.8)} & \textbf{40.2}$^{\dagger}$ &
  % \textbf{63.5}$^{\dagger}$ \\
  %  & & & 54k & Full Set & 34.1 & 28.9 \textcolor{green}{(+7.1)} & 55.2 \textcolor{green}{(+22.3)} & 28.8 & 54.9 \\
  % \cline{1-1}\cline{3-10}
  % \multirow{3}{*}{2. Reasoning Direction} & & Fwd & 25k & Correctness-Filt. & 44.9 & 42.7 \textcolor{green}{(+20.9)} & 64.7 \textcolor{green}{(+31.8)} & 40.2 & 63.5 \\
  %  & & Bwd & 25k & Correctness-Filt. & 35.4 & 39.3 \textcolor{green}{(+17.5)} & 61.3 \textcolor{green}{(+28.4)} & -- & -- \\
  %  & & Bi-directional & 25k & Correctness-Filt. & \textbf{44.3} & \textbf{45.7}$^{\dagger}$ \textcolor{green}{(+23.9)} & \textbf{67.4}$^{\dagger}$ \textcolor{green}{(+34.5)} & -- & -- \\
  % \hline
  % \end{tabular}%
  % }
  % \footnotesize
  %  \textit{Note:} All Pass@k scores are reported as percentages. $^{\dagger}$ indicates the best result across all training configurations for each metric. Abbreviations: Filt. = Filtered, perf. = performing, Fwd: Forward only CoT samples, Bwd: Backward only CoT samples.
  % \end{table*}

  
% Define colors for code-like appearance
\definecolor{promptblue}{RGB}{0, 102, 204}
\definecolor{lightgray}{RGB}{245, 245, 245}
\definecolor{lightblue}{RGB}{240, 248, 255}
\definecolor{lightgreen}{RGB}{240, 255, 240}
\definecolor{darkgray}{RGB}{64, 64, 64}
\definecolor{commentgreen}{RGB}{0, 128, 0}
\definecolor{keywordblue}{RGB}{0, 0, 255}
\definecolor{stringred}{RGB}{163, 21, 21}

% \begin{figure}[t!]
% \centering

% \begin{tcolorbox}[
%     enhanced,
%     fontupper=\ttfamily\scriptsize,
%     colback=lightgray,
%     colframe=darkgray,
%     boxrule=0.8pt,
%     arc=3pt,
%     enlarge top by=0.5mm,
%     enlarge bottom by=0.5mm,
%     title={\scriptsize\textbf{Forward Chain-of-Thought Template}},
%     coltitle=white,
%     colbacktitle=darkgray,
%     boxsep=2pt,
%     left=4pt,
%     right=4pt,
%     top=2pt,
%     bottom=2pt,
%     nobeforeafter,
%     before upper={\setlength{\parskip}{1pt}},
% ]

% \colorbox{lightblue}{\parbox{\dimexpr\linewidth-2\fboxsep}{%
% {\color{promptblue}\textbf{<Instruction>}}

% You are given a Python function. Predict what the function will output given the input.

% {\color{promptblue}\textbf{<Function>}}

% [Function Signature and Docstring]

% {\color{promptblue}\textbf{<Question>}}

% What is the output of `function\_name` given the input?
% }}

% \vspace{2pt}

% \colorbox{lightgreen}{\parbox{\dimexpr\linewidth-2\fboxsep}{%
% {\color{promptblue}\textbf{<CoT>}}
% \begin{enumerate}[leftmargin=8pt, itemsep=1pt, parsep=0pt]
% \item The function begins by initializing local variables (e.g., `counter=0`, `result=[]`).
% \item Based on the input `...`, the program enters a loop/conditional branch.  
% \item The state of variables is updated at each step, reflecting the execution trace.
% \item The final return value is computed based on the terminal state of the variables.
% \end{enumerate}

% {\color{promptblue}\textbf{<Predicted Output>}}

% [Final Output]
% }}

% \end{tcolorbox}
% {\footnotesize\textbf{(a) A forward CoT sample from input to output.}}



% \begin{tcolorbox}[
%     enhanced,
%     fontupper=\ttfamily\scriptsize,
%     colback=lightgray,
%     colframe=darkgray,
%     boxrule=0.8pt,
%     arc=3pt,
%     enlarge top by=0.5mm,
%     enlarge bottom by=0.5mm,
%     title={\scriptsize\textbf{Backward Chain-of-Thought Template}},
%     coltitle=white,
%     colbacktitle=darkgray,
%     boxsep=2pt,
%     left=4pt,
%     right=4pt,
%     top=2pt,
%     bottom=2pt,
%     nobeforeafter,
%     before upper={\setlength{\parskip}{1pt}},
% ]

% \colorbox{lightblue}{\parbox{\dimexpr\linewidth-2\fboxsep}{%
% {\color{promptblue}\textbf{<Instruction>}}

% You are given a Python function and its output. Predict what input could have produced this output.

% {\color{promptblue}\textbf{<Function>}}

% [Function Signature and Docstring]

% {\color{promptblue}\textbf{<Question>}}

% Given the function and the output, what input could have produced it?
% }}

% \vspace{2pt}

% \colorbox{lightgreen}{\parbox{\dimexpr\linewidth-2\fboxsep}{%
% {\color{promptblue}\textbf{<CoT>}}
% \begin{enumerate}[leftmargin=8pt, itemsep=1pt, parsep=0pt]
% \item To obtain the output `...`, the function must have returned from line `L`.
% \item This requires the variable `...` to hold the value `...` at that point.
% \item Tracing backwards, to achieve this state, the preceding loop must have executed `N` times.
% \item For the loop to execute `N` times, the initial input must satisfy the condition `...`.
% \end{enumerate}

% {\color{promptblue}\textbf{<Predicted Input>}}

% [Plausible Input]
% }}

% \end{tcolorbox}
% {\footnotesize\textbf{(b) A backward sample CoT to plausible input.}}
% \caption{Abstract templates for the generated bi-directional CoT data. Blue-shaded regions show the input prompt components (instruction, function, and question), while green-shaded regions show the model's reasoning and output generation.}
% \label{fig:cot-templates}
% \end{figure}




\begin{figure}[t!]
  \centering
  \footnotesize
  \begin{tabular}{@{}p{0.48\columnwidth}@{\hspace{0.01\columnwidth}|}@{\hspace{0.01\columnwidth}}p{0.48\columnwidth}@{}}
  \toprule
  \textit{Forward CoT} (input$\rightarrow$output) & \textit{Backward CoT} (output$\rightarrow$input) \\
  \midrule
  \colorbox{lightblue}{\parbox{0.95\linewidth}{\scriptsize
  \texttt{<Instruction>} Given weighted graph, find cheapest path.\\[0.05cm]
  \texttt{<Function>} [find\_cheapest\_path]\\[0.05cm]
  \texttt{<Question>} What is route from 'A' to 'F'?
  }}
  &
  \colorbox{lightblue}{\parbox{0.95\linewidth}{\scriptsize
  \texttt{<Instruction>} Given weighted graph, find start node.\\[0.05cm]
  \texttt{<Function>} [find\_cheapest\_path]\\[0.05cm]
  \texttt{<Question>} What start node gives cost 7 to 'F'?
  }}
  \\[0.2cm]
  \colorbox{lightgreen}{\parbox{0.95\linewidth}{\scriptsize
  \texttt{<CoT>}
  \begin{enumerate}[leftmargin=8pt, itemsep=0.5pt, parsep=0pt, topsep=2pt]
  \item Initialize distances: 'A'=0, others=$\infty$
  \item Extract 'A', update neighbors 'B', 'C'
  \item \dots (continue shortest path)
  \item Reach 'F', backtrack to find path
  \end{enumerate}
  \texttt{<Predicted Output>} (['A','C','E','F'], 7)
  }}
  &
  \colorbox{lightgreen}{\parbox{0.95\linewidth}{\scriptsize
  \texttt{<CoT>}
  \begin{enumerate}[leftmargin=8pt, itemsep=0.5pt, parsep=0pt, topsep=2pt]
  \item Final cost=7 to 'F', trace backwards
  \item Predecessor of 'F' is 'E' (cost 2), so cost to 'E'=5
  \item \dots (continue backtracking)
  \item Backtrack through 'C' confirms start='A'
  \end{enumerate}
  \texttt{<Predicted Input>} 'A'
  }}
  \\
  \bottomrule
  \end{tabular}
  \vspace{-0.1cm}
  \caption{Bi-directional CoT data format examples. Blue regions show input prompt (instruction, function, question); green regions show model's trace-grounded reasoning and prediction. Forward samples reason from input to output, backward samples deduce input from output.}
  \label{fig:cot-templates}
  \end{figure}

\begin{figure}[t!]
  \centering
  \footnotesize
  \begin{tabular}{@{}p{0.48\columnwidth}@{\hspace{0.01\columnwidth}|}@{\hspace{0.01\columnwidth}}p{0.48\columnwidth}@{}}
  \toprule
  \textit{Function Signature} & \textit{Class Signature} \\
  \midrule
  \ttfamily\scriptsize
  Function: solution(\\
  \hspace*{1em}freq\_list: \\
  \hspace*{2em}list[tuple[\\
  \hspace*{3em}str, int]]) \\
  \hspace*{1em}-> dict[str, str]
  &
  \ttfamily\scriptsize
  Class: HuffmanTree; \\
  \_\_init\_\_(self, \\
  \hspace*{1em}freq\_list: \\
  \hspace*{2em}list[tuple[\\
  \hspace*{3em}str, int]]) \\
  \hspace*{1em}-> None; \\
  build\_tree(self) \\
  \hspace*{1em}-> tuple; \\
  get\_encoding(self) \\
  \hspace*{1em}-> dict[str, str]
  \\
  \bottomrule
  \end{tabular}
  \vspace{-0.1cm}
  \caption{Signature format templates enforced via few-shot examples to ensure type and naming consistency across candidate solutions and tests.}
  \label{fig:signature-formats}
\end{figure}

\begin{figure}[t!]
  \centering
  \footnotesize
  \begin{tabular}{@{}p{0.48\columnwidth}@{\hspace{0.01\columnwidth}|}@{\hspace{0.01\columnwidth}}p{0.48\columnwidth}@{}}
  \toprule
  \textit{Correct Format} \textcolor{correctgreen}{\cmark} & \textit{Prohibited Format} \textcolor{errorred}{\xmark} \\
  \midrule
  \colorbox{correctgreen!10}{\parbox{0.95\linewidth}{\ttfamily\scriptsize
  def test\_basic(): \\
  \hspace*{1em}\# Test basic \\
  \hspace*{1em}assert \\
  \hspace*{2em}solution([1, \\
  \hspace*{3em}2, 3], 2) \\
  \hspace*{2em}== [1]
  }}
  &
  \colorbox{errorred!10}{\parbox{0.95\linewidth}{\ttfamily\scriptsize
  def test\_wrong(): \\
  \hspace*{1em}\# Variable \\
  \hspace*{1em}\# outside assert \\
  \hspace*{1em}lst = [1, 2, 3] \\
  \hspace*{1em}assert \\
  \hspace*{2em}solution(lst, \\
  \hspace*{3em}2) == [1]
  }}
  \\
  \bottomrule
  \end{tabular}
  \vspace{-0.1cm}
  \caption{Test format requirements. Correct format (left) enables clean I/O extraction for trace generation with direct function calls in assert statements, while prohibited format (right) complicates trace analysis with intermediate variable assignments.}
  \label{fig:test-formats}
\end{figure}